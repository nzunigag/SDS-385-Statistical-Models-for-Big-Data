\documentclass[11 pt]{article} 
\usepackage[left=2cm, top=2cm, right=2cm, bottom=2.5cm, footskip=.5cm]{geometry}
\usepackage{url}
\usepackage{enumitem}

\author{Natalia Zuniga-Garcia}
\title{Peer Review 2}
\date{November 3, 2017}

\begin{document}

\maketitle

\section{Description}

I went trough the documentation available in the student's GitHub page by November 3, 2017.

\begin{itemize}
	\item Student evaluated:  Yanxin Li (\url{https://github.com/Cindy-UTSDS/BigData})
	\item Assignment reviewed: Exercises 2 to 6


\end{itemize}
	
\section{Code Style Evaluation}
In this section I provide comments regarding the main code style characteristics: notation and naming, syntax, organization, and function definition. I based my evaluation criteria mainly on the documentation provided by the professor``Advanced R" by Hadley Wickham \footnote{\url{http://adv-r.had.co.nz/Style.html}}, and ``Google's R Style Guide"\footnote{\url{https://google.github.io/styleguide/Rguide.xml}}. 

\subsection{Notation and Naming}
\begin{itemize}
	\item \textbf{File names}: I would suggest that the files' name do not include a space. You can use an hyphen (-) or a underscore (\_) to separate the name of the files instead of a space. I saw you make this while naming your PDF and R code solutions.
	\item \textbf{Object names}: I like the way you named the objects and the functions. It makes it easier to follow the code.
\end{itemize}

\subsection{Syntax}
\begin{itemize}
	\item \textbf{Spacing}: In general, the code presents adequate spacing. 
	\item \textbf{Curly braces}: You used appropriately the curly braces. 
	\item \textbf{Line length}: There are some cases where you have long lanes, mainly when you comment the operations. I would suggest to make the comment in the previous or posterior lane to avoid this. Also, if you have a long code line you can break the line.
	\item \textbf{Indentations}: The indentation of your code seems appropriate to me.
	\item \textbf{Assignment}: You correctly used $<-$ to the assignments.
\end{itemize}

\subsection{Organization}
Trough the code you provided adequate comments that guided me easily. You used \#---------------- to separate your code into chunks, I like it because it makes it easy to read part by part.  I started to use this format because of you, thanks!.

\subsection{Function definition}
I would suggest to comment more when defining a function. You can comment more about what the input and output is and about what object types are. You did this for some functions and not for others. I would encourage you to use it every time you define a function. 

\section{General Comments}
\begin{itemize}
	\item In Exercises 2. I like your code here. It is very easy to understand and follow. I think your implementation of the Stochastic Gradient Descent is very straightforward and you explained every step. 
	\item In Exercises 2. You provided a detailed explanation of the solution in your PDF file. This time you added multiple graphs to explain your results. It is very useful for the reader and I think you did a very good job by providing these results in a graphical way.
	\item In Exercises 3. You explained very well both Line Search and Quasi-Newton Methods. I really learned a lot from you in this part. Your implementation and your explanation is very clear and you presented the results with plots that help to understand better.
	\item In Exercises 5. I saw that, beyond these exercises,  you started to explain more in your GitHub page. This is good because if someone else (not from the course) review your page you provided enough information to explain your work. 
	\item In Exercise 5. You also published your code in \textit{rpubs.com} (http://rpubs.com/leexiner/bigdata-exercise05) this is very good as well. I didn't know you can do that now I will try to use this resource! 
	\item In Exercise 5. You also explained your work very thoroughly in the PDF document. The graphs you provided are very illustrative and help to understand the concepts.  
	\item In Exercise 6. You also published your code and explained very well your GitHub page documentation. 
	\item In Exercise 6. Your mathematical derivations are very well explained and help me to understand better.
	\item The fact that you used Latex to describe your mathematical derivation makes your documentation look very professional.
	\item I think the way you present your code at the end of some of your the documents is very good. It make the work look more professional as well.
\end{itemize}

\section{Summary}
Dear Yanxin, 

I hope that my comments help you to improve even more! I think you have evolved a lot since the last Peer Review. Your GitHub Page looks very organized and you started to explain every exercise on you page. You also started publishing your code in \textit{rpubs.com} which makes it easier to follow from your GitHub page. Your explanations of the results are very useful, you started to provide more graphical explanations and your documents look very professional. I also learned a lot from you this time! 
Good Job!


\end{document}